\documentclass{beamer}

\renewcommand\sfdefault{phv}
\renewcommand\familydefault{\sfdefault}
\usetheme{default}
\usepackage{color}
\useoutertheme{default}
\usepackage{texnansi}
\usepackage{color}
\usepackage{marvosym}
\definecolor{bottomcolour}{rgb}{0.32,0.3,0.38}
\definecolor{middlecolour}{rgb}{0.08,0.08,0.16}
\setbeamerfont{title}{size=\Huge}
\setbeamercolor{structure}{fg=gray}
\setbeamertemplate{frametitle}[default]%[center]
\setbeamercolor{normal text}{bg=black, fg=white}
\setbeamertemplate{background canvas}[vertical shading]
[bottom=bottomcolour, middle=middlecolour, top=black]
\setbeamertemplate{items}[circle]
\setbeamerfont{frametitle}{size=\huge}
\setbeamertemplate{navigation symbols}{} %no nav symbols


\usepackage{amsmath,  amsfonts, amsthm, graphicx, subfigure}
%\usepackage{biblatex}
 \usepackage{fancybox, ulem}
 \usepackage{mathtools}
 \usepackage{tabularx}
 \usepackage{tikz}
 \usepackage{movie15}
 %\bibliography{pumping_paper}
\newcommand{\p}{\partial}
\newcommand{\f}{\frac}
\newcommand{\B}{\textbf}
\newcommand{\I}{\textit}
\newcommand{\tab}{\hspace{10mm}}

 % \usetheme{Singapore}
% \usetheme{Warsaw}
  \setbeamertemplate{navigation symbols}{}
\title{ Matlab Notes Part 3}
\author{Austin Baird\\UNC Department of Mathematics\\UNC Department of Biology}
\date{\today} 

\begin{document}
\frame{\titlepage}

\begin{frame}
\frametitle{Summary}
\begin{itemize}

\item Last week we learned: 
\begin{itemize}
\item For loops and control sequences.
\item Functions: anonymous and traditional functions.
\item Using these functions. 
\item writing our first function 
\end{itemize}
\item Today we will: 
\begin{itemize}
\item Learn how to use built in Matlab functions.
\item Systems of equations and how they relate to matrix functions.  
\item Understanding the output.
\item Integrating them into functions.
\end{itemize}
\end{itemize}

\end{frame}

%--------------------------------------------------------------------------------------------------------------------------------------------------------------------------------------------------------------------------%
\begin{frame}
\frametitle{Goal of Today} 

We want to be able to analyze a system of ODE's numerically and produce an output that makes sense using matlab built in functions to solve systems of ODEs: 

\begin{align*}
\frac{dy_1}{dt} &= f(y_1,y_2,...,t)\\
\frac{dy_2}{dt} &= f(y_1,y_2,...,t)\\
\frac{dy_3}{dt} &= f(y_1,y_2,...,t)\\
...\\
\frac{dy_n}{dt} &= f(y_1,y_2,...,t)\\
\end{align*}


\end{frame}
%
%%--------------------------------------------------------------------------------------------------------------------------------------------------------------------------------------------------------------------------%
\begin{frame}
\frametitle{Systems of ODEs}

Note that this is equivalent to writing the system in the form: 

{\LARGE
\begin{align*}
\frac{d\hat{y}}{dt} = F(\hat{y},t)
\end{align*}
}

Which in the end is a matrix equation. We will discuss this at length now! 
\end{frame}


%--------------------------------------------------------------------------------------------------------------------------------------------------------------------------------------------------------------------------%
\begin{frame}
\frametitle{Systems of Equations}

Because Matlab is a matrix manipulator we want to see the connection between a system of equations and writing it as a matrix system (Ax=b).

\begin{align*}
a_{11}x_1 + a_{12}x_2 + ... a_{1n}x_n &= y_1\\  
a_{21}x_1 + a_{22}x_2 + ... a_{2n}x_n &= y_2\\  
...\\
a_{m1}x_1 + a_{m2}x_2 + ... a_{mn}x_n &= y_m\\
\end{align*}  


\end{frame}
%%--------------------------------------------------------------------------------------------------------------------------------------------------------------------------------------------------------------------------%

\begin{frame}
\frametitle{Systems of Equations}

The system of equations on the previous page can now be written as:
\ \\
\ \\
$$
%A_{m,n} =
\begin{pmatrix}
  a_{1,1} & a_{1,2} & \cdots & a_{1,n} \\
  a_{2,1} & a_{2,2} & \cdots & a_{2,n} \\
  \vdots  & \vdots  & \ddots & \vdots  \\
  a_{m,1} & a_{m,2} & \cdots & a_{m,n}
  \end{pmatrix}
%\times
\begin{pmatrix}
 x_1\\
 x_2\\
 \vdots\\
 x_n
\end{pmatrix}
= 
\begin{pmatrix}
y_1\\
y_2\\
\vdots\\
y_m
\end{pmatrix}
 $$
 
 We can now solve this by solving: $ A_{m,n} X_n = Y_m$
\end{frame}
%%%--------------------------------------------------------------------------------------------------------------------------------------------------------------------------------------------------------------------------%

\begin{frame}
\frametitle{Solving the System}

There are many ways to solve this system! The most obvious way is using the power of Matlab!

\begin{align*}
 \gg X = inv(A) Y
 \end{align*}
Done! Or we can do! 

\begin{align*}
 \gg X = A\backslash Y
 \end{align*}
 
Check to make sure these do the same thing! 

\end{frame}
%%%--------------------------------------------------------------------------------------------------------------------------------------------------------------------------------------------------------------------------%
%
\begin{frame}
\frametitle{Functions in Matlab} 


\end{frame} 
%
%%
%%%--------------------------------------------------------------------------------------------------------------------------------------------------------------------------------------------------------------------------%
\begin{frame}
\frametitle{Solving Systems of ODEs with Matlab}

Now that we've seen how to simply solve systems of equations, lets solve systems of ODEs. Usually this can be done using built in solvers ( We will go over the algorithms in the future):\\
\ \\ 

$\gg$ $[t_{out},y_{out}]$ = ode**(@F,time,$y_0$,options]\\
\ \\
Generally we want to solve $y' = f(y,t)$ over a given time domain. 
\begin{itemize}
\item Here $t_{out}$ and time are arrays of time values, 
\item @F is $f(y,t)$, which can be an array if there is a system of ODEs (example code)
\item $y_0$ are the initial conditions .  
\item $y_{out}$ is an array of values which correspond to the solution at each $dt$ (discretized time domain). 
\end{itemize}

\end{frame}





%%--------------------------------------------------------------------------------------------------------------------------------------------------------------------------------------------------------------------------%
\begin{frame}
\frametitle{Let's Try it} 

First we will try and solve the ODE: $\frac{dy}{dx} = xy^2 + y$, to do this we will define a function:\\
\ \\
F= @(x,y) $xy^2$ + y \\
\ \\
We can then solve it! [x,y] = ode45(f,[0,0.5],1)\\
Then plot the output: \\
plot(x,y)\\
\ \\
Note: If you allow the domain of this function to reach 1, then the error will explode! (why?)
\end{frame}

%%--------------------------------------------------------------------------------------------------------------------------------------------------------------------------------------------------------------------------%
\begin{frame}
\frametitle{Solving a system}

To solve a system of ODEs we can store them in an array with a separate function, then call that function when using ode45. \\
\ \\
function f = odefun(t,y)\\
\ \\
Define your parameters here...\\
\ \\ 
We now want to store the functions in an array: \\
\ \\
f(1) = y(1)+y(2)\\
f(2) = y(1)+y(2)\\ 
\ \\
or...\\
\ \\
f = [y(1) + y(2) ; y(1) + y(2)],   \textcolor{cyan}{Note:} y(1) = x and y(2) = y. If you had a z, y(3) = z. 

\end{frame}
%%--------------------------------------------------------------------------------------------------------------------------------------------------------------------------------------------------------------------------%
\begin{frame}
\frametitle{Finishing it all up!} 
Now that we have a function which stores the functions corresponding to our system of odes we call our function: \\
\ \\
initial conditions = array corresponding to [y(1) y(2) ...]\\ 
\ \\
time values = array corresponding to [time(start) time(end)]\\
\ \\
now we call the function: 
[t,y] = ode45(@f,time,initial)\\
\ \\
Now the solutions are stored as the columns of y. They can be called by: y(:,1), y(:,2)...
\end{frame}
%
%%--------------------------------------------------------------------------------------------------------------------------------------------------------------------------------------------------------------------------%
\begin{frame}
\frametitle{Homework 3}

Lorenz equations were initially developed as a weather model and mathematically have served as an example of chaotic behavior seen in non-linear systems. The Lorenz equations are defined to be: \\
\begin{align*}
\frac{dx}{dt} &= \sigma(y-x)\\
\frac{dy}{dt} &= x(\rho-z)-y\\
\frac{dz}{dt} &= xy- \beta z
\end{align*}

\textcolor{cyan}{Note:} $\sigma, \rho, \beta$ are constants and can be defined however you want them to. (i.e make them all 10) \\
I want you to solve these equations using ode45 and to create a few plots:
\end{frame}
%%--------------------------------------------------------------------------------------------------------------------------------------------------------------------------------------------------------------------------%

\begin{frame}
\frametitle{Homework 3} 
items to do: 

\begin{itemize}
\item Alter the functions I gave you so that they represent the lorenz equations.
\item Make a script which plots the strange attractor: plot(x,z) on the same graph. 
\item Have this script also plot x, y, z, over the time domain, in the same figure (using subplots) 
\item Have it make a graph of x for one set of initial conditions = [$x_0 y_0 z_0$] Then on the same figure a plot with these initial conditions: 
[$(x_0+0.1) (y_0+ 0.1) (z_0 + 0.1)$]. 

\textcolor{cyan}{Note:} Be sure to keep all parameter values greater than zero, other than that you can choose them as you like. Some are more interesting than others...
\end{itemize}
\end{frame}


\end{document}
