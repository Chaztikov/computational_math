\documentclass{beamer}

\renewcommand\sfdefault{phv}
\renewcommand\familydefault{\sfdefault}
\usetheme{default}
\usepackage{color}
\useoutertheme{default}
\usepackage{texnansi}
\usepackage{color}
\usepackage{marvosym}
\definecolor{bottomcolour}{rgb}{0.32,0.3,0.38}
\definecolor{middlecolour}{rgb}{0.08,0.08,0.16}
\setbeamerfont{title}{size=\Huge}
\setbeamercolor{structure}{fg=gray}
\setbeamertemplate{frametitle}[default]%[center]
\setbeamercolor{normal text}{bg=black, fg=white}
\setbeamertemplate{background canvas}[vertical shading]
[bottom=bottomcolour, middle=middlecolour, top=black]
\setbeamertemplate{items}[circle]
\setbeamerfont{frametitle}{size=\huge}
\setbeamertemplate{navigation symbols}{} %no nav symbols


\usepackage{amsmath,  amsfonts, amsthm, graphicx, subfigure}
%\usepackage{biblatex}
 \usepackage{fancybox, ulem}
 \usepackage{mathtools}
 \usepackage{tabularx}
 \usepackage{tikz}
 \usepackage{movie15}
 %\bibliography{pumping_paper}
\newcommand{\p}{\partial}
\newcommand{\f}{\frac}
\newcommand{\B}{\textbf}
\newcommand{\I}{\textit}
\newcommand{\tab}{\hspace{10mm}}

 % \usetheme{Singapore}
% \usetheme{Warsaw}
  \setbeamertemplate{navigation symbols}{}
\title{Lecture 5}
\author{Austin Baird\\UNC Department of Mathematics\\UNC Department of Biology}
\date{\today} 

\begin{document}
\frame{\titlepage}

\begin{frame}
\frametitle{Summary}
\begin{itemize}

\item Last week we learned: 
\begin{itemize}
\item Numerical integration.
\item Rectangle rule.
\item Trapizoid rule
\item Review of matlab solvers used in solving non-linear ODEs
\end{itemize}
\item Today we will: 
\begin{itemize}
\item Runge-Kutta.
\item Theory and implementation 
\item Error analysis of ode solvers
\end{itemize}
\end{itemize}

\end{frame}

%--------------------------------------------------------------------------------------------------------------------------------------------------------------------------------------------------------------------------%
\begin{frame}
\frametitle{Problem formulation}

We want to be able to evaluate equations of the form: 

\begin{align*}
\frac{dy}{dt} &= f(t,y)\\
y(t_0) &= y_0
\end{align*}

\ \\

Here $t_0$ is initial time, and $y_0$ is the initial value. We've done this once before with eulers method, but is that good enough? 
\end{frame}


%%%--------------------------------------------------------------------------------------------------------------------------------------------------------------------------------------------------------------------------%
\begin{frame}
\frametitle{Error and Doing a Bit Better}

Recall from our previous homework that Eulers method is a first order method: $err = \mathcal{M}h$. Or that error is a linear function of step size $ h=\Delta t$. Also recall 
Eulers method: 

\begin{align*}
y(t_{n+1})& \approx y_{n} + hf(t_n,y_n)\\
&or\\
y_{n+1} &= y_{n} + hf(t_n,y_n)
\end{align*}

here we approximate the solution, $y(t_{n+1})$ of the n+1 time step with previously computed data: $y_n$ and the derivative: $f(t_n,y_n)$ 


\end{frame}


%%--------------------------------------------------------------------------------------------------------------------------------------------------------------------------------------------------------------------------%
\begin{frame}
\frametitle{Error and Doing a Bit Better}

Because Euler's method is a first order method, if we want four digits of accuracy, then we have to select a time step: 
\begin{align*}
\Delta t = 0.0001 
\end{align*}

Say we want to compute the solution of ten seconds: 

\begin{align*} 
N_{timesteps} = \frac{10}{0.0001} = 100000
\end{align*}

One hundred thousand time steps! Seems a bit steep and can take a significant amount of time to compute in practice. 
\end{frame}
%%%--------------------------------------------------------------------------------------------------------------------------------------------------------------------------------------------------------------------------%

\begin{frame}
\frametitle{Error and Doing a Bit Better} 

Now say that we have a second order method: 

\begin{align*}
err = \mathcal{M}h^2
\end{align*}

Then we can compute the time step needed to get four digits of accuracy: 

\begin{align*}
\sqrt{0.0001} = 0.01 = \Delta t
\end{align*}

We can now see that it will only take 1000 times steps to compute the solution over 10 seconds. So much better!! 
 
\end{frame}
%%%%--------------------------------------------------------------------------------------------------------------------------------------------------------------------------------------------------------------------------%
%
\begin{frame}
\frametitle{log-log plots} 

We can now extract the order of a given method by doing the following: \\
\ \\
If we have the error = $err$ and the time step: $\Delta t$, then we have the following relation (which can be seen in graph form): 

\begin{align*}
err = \mathcal{M}h^2
\end{align*}

This means that the error is a quadratic function of time step $h = \Delta t$. We can also take the log, $ln$, of the data and get: 
\begin{align*}
log(err) = 2log(h) + log(\mathcal{M})
\end{align*}

This shows now that we have a linear function with slope 2. If we do this to our data and take the slope of the line, then we can get the order of the method! 

 
\end{frame}

%%%%--------------------------------------------------------------------------------------------------------------------------------------------------------------------------------------------------------------------------%


\begin{frame}
\frametitle{Example of a Second Order Method}

Now we want to use the value of the derivative $f(t,y)$ at an intermediate time step: $\frac{t}{2}$. We want to use the value of the derivative between $t_n$ and $t_{n+1}$. We will first introduce the \textcolor{cyan}{Midpoint Method} or also known as \textcolor{cyan}{Two step Runge-Kutta}: 

\begin{align*}
k_1 &= hf(t_n,y_n)\\
k_2 &= hf(t_n + \frac{1}{2}h, y_n + \frac{1}{2}k_1)\\
y_{n+1} &= y_n + k_2
\end{align*}

This is whats known as a \textcolor{cyan}{Multi-step Method} and it gives us second order accuracy (We need to show this numerically!). 

\end{frame}
%%%%--------------------------------------------------------------------------------------------------------------------------------------------------------------------------------------------------------------------------%
%%
%%%--------------------------------------------------------------------------------------------------------------------------------------------------------------------------------------------------------------------------%
\begin{frame}
\frametitle{Matlab Implimentation}

What do we need to compute this? In general all the tools we used in Eulers method apply here, with one additional computational step. 

\begin{itemize}
\item Discretize the domain, $t = [t_0,t_1,\hdots,t_n]$
\item Use the function: $f(t,y(t))$ 
\item First we compute $k_1$ 
\item $\gg k_1 =  hf(t_n,y_n)$
\item Now use this value in the next computational step: 
\item $\gg k_2 = hf(t_n + \frac{1}{2}h, y_n + \frac{1}{2}k_1)$
\item Then we can compute our  new approximation: 
\item $\gg y_{n+1} = y_n + k_2$
\item Repeat at each time step! 
\end{itemize}


\end{frame}
%%%--------------------------------------------------------------------------------------------------------------------------------------------------------------------------------------------------------------------------%

\begin{frame}
\frametitle{Homework 5} 
We know that Euler method is 1st order, we want to verify that this new method is second order. 
\begin{itemize}
\item Alter your Euler code so that it deals with time derivatives (this should be just changing x's to t's) 
\item Write a function which will solve $ \frac{dy}{dt} = f(t,y)$ Using the two step Runge-Kutta method (introduced in these lecture notes).
\item Write a separate script which does the following: 
\begin{itemize}
\item A plot which graphs error v $\Delta t$. 
\item A log-log  plot of $ err$ v $\Delta t$. ( you can google log-log plot matlab, or you can just take $\gg ln(err), \gg ln(\Delta t)$, and graph.
\item A print out of the slope of the line that you just graphed (this should be close to 2, second order method!)
\item A plot of the graph of (eulers method - midpoint method) v time
\item A print out of the max error over the entire time simulation between the two methods. (i.e. max(euler-midpoint) )
\end{itemize}
\end{itemize}
\end{frame}

%
\end{document}
