\documentclass{beamer}

\renewcommand\sfdefault{phv}
\renewcommand\familydefault{\sfdefault}
\usetheme{default}
\usepackage{color}
\useoutertheme{default}
\usepackage{texnansi}
\usepackage{color}
\usepackage{marvosym}
\definecolor{bottomcolour}{rgb}{0.32,0.3,0.38}
\definecolor{middlecolour}{rgb}{0.08,0.08,0.16}
\setbeamerfont{title}{size=\Huge}
\setbeamercolor{structure}{fg=gray}
\setbeamertemplate{frametitle}[default]%[center]
\setbeamercolor{normal text}{bg=black, fg=white}
\setbeamertemplate{background canvas}[vertical shading]
[bottom=bottomcolour, middle=middlecolour, top=black]
\setbeamertemplate{items}[circle]
\setbeamerfont{frametitle}{size=\huge}
\setbeamertemplate{navigation symbols}{} %no nav symbols


\usepackage{amsmath,  amsfonts, amsthm, graphicx, subfigure}
%\usepackage{biblatex}
 \usepackage{fancybox, ulem}
 \usepackage{mathtools}
 \usepackage{tabularx}
 \usepackage{tikz}
 \usepackage{movie15}
 %\bibliography{pumping_paper}
\newcommand{\p}{\partial}
\newcommand{\f}{\frac}
\newcommand{\B}{\textbf}
\newcommand{\I}{\textit}
\newcommand{\tab}{\hspace{10mm}}

 % \usetheme{Singapore}
% \usetheme{Warsaw}
  \setbeamertemplate{navigation symbols}{}
\title{ Matlab Notes Part 2}
\author{Austin Baird\\UNC Department of Mathematics\\UNC Department of Biology}
\date{\today} 

\begin{document}
\frame{\titlepage}

\begin{frame}
\frametitle{Summary}
\begin{itemize}

\item Last week we learned: 
\begin{itemize}
\item How to do basic arithmetic
\item plotting 
\item Discretizing the real line
\item writing our first function 
\item outputting text to the screen 
\item (HW) Plotting multiple items on the same figure
\end{itemize}
\item Today we will: 
\begin{itemize}
\item Learn control sequences
\item How to use built in matlab functions
\item More plotting! 
\item How to create your own function in matlab
\end{itemize}
\end{itemize}

\end{frame}

%--------------------------------------------------------------------------------------------------------------------------------------------------------------------------------------------------------------------------%
\begin{frame}
\frametitle{Control Sequences} 

\begin{itemize}
\item If we want to automate a task these tools are how we do it! 
\item (for, while, if) are the words used to describe the different types of control we have over a given sequence of commands. 
\item In general a for loop is executed as follows: 
\end{itemize}

for (conditionals, generally a sequence of numbers), \\
\ \\


	\hspace{10mm} body to be executed \\
	\ \\
	
end
\end{frame}
%
%%--------------------------------------------------------------------------------------------------------------------------------------------------------------------------------------------------------------------------%
\begin{frame}
\frametitle{Examples}

Here is pseudo code which counts! \\
\ \\
i=0
\ \\
for i<10, \\

\hspace{10mm} print i \\
\hspace{10mm} i = i+1 \\

end
\ \\
\ \\
What does this do? Now try it! 
\end{frame}


%--------------------------------------------------------------------------------------------------------------------------------------------------------------------------------------------------------------------------%
\begin{frame}
\frametitle{Nested for loops} 

Say we are given a matrix and we want to step through the values. Given: mxn matrix A : \\
\ \\
for i = 1: m,\\
\tab for j = 1:n, \\
\tab \tab disp(A(i,j))\\
\tab end\\
end\\
\ \\
This will print out each row of the matrix A, how do we print out the columns? 

\end{frame}
%--------------------------------------------------------------------------------------------------------------------------------------------------------------------------------------------------------------------------%

\begin{frame}
\frametitle{Examples of other types } 

If statements only execute the body ``if'' a certain criteria is met: 

\ \\
i=0\\
\ \\
for i<10,\\
\tab if i==2,\\
\tab \tab print i \\
\tab end \\

\tab i = i+1 \\
\ \\
end

\ \\
\ \\
See what this does, if statements can have a wide variety of conditionals, use your imagination! 



\end{frame}
%%--------------------------------------------------------------------------------------------------------------------------------------------------------------------------------------------------------------------------%

\begin{frame}
\frametitle{Using ``While'' loops}

I would say that while loops are most useful in scientific computing to execute the body till a certain criteria is met:\\
\ \\
\ \\
while err > 0.1, \\
\ \\
\tab Execute root finding algorithm\\
\tab Need expression for error here\\
\ \\
end 

Lets try and use a while loop to make a root finding algorithm!
\end{frame}
%%--------------------------------------------------------------------------------------------------------------------------------------------------------------------------------------------------------------------------%
%
\begin{frame}
\frametitle{Functions in Matlab} 

There are two ways to define functions in matlab: 
\begin{itemize}
\item ``anonymous'' functions (which can be used to define analytical functions in scripts) 
\item High level functions which require their own script. 
\end{itemize}
 ``anonymous'' functions: 

\tab \tab f = @(arglist)expression\\

Example: 

\tab \tab y = @(x) $x^2$ \\

This produces a function (y) which takes in a domain (x) and outputs $x^2$ to the variable y. \\
Make sense? Best if we try it out! 


\end{frame} 
%
%
%%--------------------------------------------------------------------------------------------------------------------------------------------------------------------------------------------------------------------------%
\begin{frame}
\frametitle{Functions}
There are programs which can take a given number of inputs and output what you want them to (very useful). \\
You must ``feed'' it all the inputs you want it to compute with because any outside variables will not be passed into it: \\
\ \\
$function[outputs] = function name(inputs)$\\
\ \\
\tab Body to be executed \\
\ \\
end\\
\ \\
\ \\

Note that functions can have multiple functions in the same script and can use all of the self contained programs! 

\end{frame}





%--------------------------------------------------------------------------------------------------------------------------------------------------------------------------------------------------------------------------%
\begin{frame}
\frametitle{Let's Try it} 

Create a function which takes a number, n as its input and counts up to n. 
\ \\
\ \\
\pause
function[] = function\_count(n)\\
\ \\
\tab for i = 1:n, \\
\tab \tab disp(i) \\
\tab end\\
\ \\
end

We can use this function by making sure we are in the same directory as it (current folder in Matlab). \\
then creating a number $n = 10$ on the command line \\
typing function\_count(n)

\end{frame}

%--------------------------------------------------------------------------------------------------------------------------------------------------------------------------------------------------------------------------%
\begin{frame}
\frametitle{Euler's Method}
We want to write a function to solve (approximate) the solution to: \\

\begin{align*}
{y}^{\prime}(x) &= f(x,y)\\
y_0 &= y(x_0)
\end{align*}

Given an initial point of the solution curve $(x_0, y_0)$ and a step size $h=dx$ we can approximate the solution to this equation at $x_0, x_1 = x_0 + h, x_2 = x_0 + 2h ...$. 

\begin{align*}
y_1 &= y_0 + hf(x_0,y_0)\\
y_2 &= y_1 + hf(x_1,y_1)...
\end{align*}
\end{frame}
%--------------------------------------------------------------------------------------------------------------------------------------------------------------------------------------------------------------------------%
\begin{frame}
\frametitle{Planning to write a function} 

What items do we need to create a function which will approximate this ODE  using Eulers Method? 

\begin{itemize}
\item Step size h
\item initial value $(x_0,y_0)$
\item The right hand side of the equation: $f(x,y)$
\item A discretized domain, $x$
\item An array which is size(x) to store the approximate values of our function
\end{itemize}

Then the question to ask is if I were to make a function which of these would I want to input into the function? (up to you!)

\begin{itemize}
\item Step size h 
\item initial value $(x_0,y_0)$
\item The right hand side of the domain, $f(x,y)$
\item (maybe) discretized domain and empty array?
\end{itemize}

\end{frame}

%--------------------------------------------------------------------------------------------------------------------------------------------------------------------------------------------------------------------------%
\begin{frame}
\frametitle{Homework 2}

 1) Make a function which uses Euler's method to approximate a given ODE: 
\begin{align*}
y^{\prime}(x) = f(x,y)
\end{align*}

It MUST:
\begin{itemize}
\item It must take in at least three variables. 
\item To be handed in with the code (separate document):

\begin{itemize}
\item print out the maximum error between the approximate solution and the actual solution (ex: make $ f(x,y)=y$, then you have an actual solution 
\item A graph of the error as a function of step size, h
\item graph the actual solution and the approximate solution over a given domain (ex x=[0,2]) 
\end{itemize}
\end{itemize}

2) Make a function which takes in a matrix and outputs its upper triangular form (use nested for loops, not built in matlab functions)
\end{frame}
%--------------------------------------------------------------------------------------------------------------------------------------------------------------------------------------------------------------------------%

\begin{frame}
\frametitle{Homework 2} 

Also read chapter 5 of the Matlab book. It will help a lot with the homework. Also the wikipedia article on Eulers method can be helpful... 

\end{frame}


\end{document}
